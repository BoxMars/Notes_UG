\documentclass{beamer}
\usepackage{amsmath}
\usepackage{amssymb}
\usepackage{amsthm}                                        
\usepackage{textcomp}     
\usepackage{xcolor}
\usepackage{listings} 
\usetheme{Madrid}
\title{Quaternion Convolutional Neural Networks}
\author{Xuanyu Zhu\and Yi Xu \and  Hongteng Xu\and Changjian Chen}
\date{\today}
\begin{document}

\begin{frame}
    \titlepage
\end{frame}

% \begin{frame}
%     \frametitle{Table of Content}
%     \tableofcontents
% \end{frame}

\section{Introduction}
\begin{frame}
    \frametitle{Introduction}

    As a powerful feature representation method, convolutional neural networks (CNNs) have been widely applied in the field of computer vision. 
    \\~\\
    One key module of CNN model is the convolution layer, which extracts features from high-dimensional structural data efficiently by a set of convolution kernels. 
    \\~\\
    When dealing with multi-channel inputs (e.g., color images), the convolution kernels merges these channels by summing up the convolution results and output one single channel per kernel accordingly.
\end{frame}


\begin{frame}
    \frametitle{Introduction}
    \begin{figure}[H]
        \centering
        \includegraphics[width=.7\textwidth]{img/pic1.jpg}
        \caption{Real-valued CNN}
    \end{figure}
\end{frame}

\begin{frame}
    \frametitle{Introduction}

    We may lose important structural information of color and obtain non-optimal representation of color image.
    \\~\\
    Simply summing up the outputs gives too many degrees of freedom to the learning of convolution kernels.
    \\~\\
    We may have a high risk of over-fitting even if imposing heavy regularization terms.
\end{frame}

\begin{frame}
    \frametitle{Introduction}

    We propose a novel quaternion convolutional neural network (QCNN) model, which represents color image in the quaternion domain.
    \\~\\
    While the traditional real-valued convolution is only capable to enforce scaling transformation on the input, specifically, the quaternion convolution achieves the scaling and the rotation of input in the color space, which provides us with more structural representation of color information.
    \\~\\
    We establish fully-quaternion CNNs to represent color images in a more effective way and study the relationship between our QCNN model and existing real-valued CNNs and find a compatible way to combine them together in a same algorithmic framework.
\end{frame}

\begin{frame}
    \frametitle{Introduction}
    \begin{figure}[H]
        \centering
        \includegraphics[width=.7\textwidth]{img/pic2.jpg}
        \caption{Quaternion CNN}
    \end{figure}
\end{frame}

\section{Related works}
\subsection{Quaternion-based color image processing}
\begin{frame}{Related works}{Quaternion-based color image processing}
    A quaternion $\hat{q}$ in the quaternion domain $\mathbb{H}$, \emph{i.e}, $q\in \mathbb{H}$, can be represented as $\hat{q}=q_0+q_1i+q_2j+q_3k$, where $q_i\in \mathbb{R}$ for $i=0,1,2,3$, and the imaginary units $i,j,k$ obey the quaternion rules that $i^2=j^2=k^2=ijk=-1$. Similar to real numbers, we can define a series of operations for quaternions:
\end{frame}

\begin{frame}{Related works}{Quaternion-based color image processing}
    \begin{itemize}
        \item Addition:
        \begin{equation*}
            \begin{split}
                \hat{p}+\hat{q}=&(p_0+q_0)+(p_1+q_1)i\\
                                +&(p_2+q_2)j+(p_3+q_3)k
            \end{split}
        \end{equation*}
        \item Scalar multiplication: $$\lambda \hat{q}=\lambda q_0+\lambda q_1i+\lambda q_2j+\lambda q_3k$$
        \item Element multiplication: 
        \begin{equation*}
            \begin{split}
                \hat{p}\hat{q}=&(p_0q_0-p_1q_1-p_2q_2-p_3q_3)\\
                            +&(p_0q_1+p_1q_0+p_2q_3-p_3q_2)i\\
                            +&(p_0q_2-p_1q_3-p_2q_0-p_3q_1)j\\
                            +&(p_0q_3+p_1q_2-p_2q_1-p_3q_0)k
            \end{split}
        \end{equation*}
    \end{itemize}
\end{frame}

\begin{frame}{Related works}{Quaternion-based color image processing}
    \begin{itemize}
        \item Conjugation: $$\hat{q}^*=q_0-q_1i-q_2j-q_3k$$
    \end{itemize}
    These quaternion operations can be used to represent rotations in a three-dimensional space. Suppose that we rotate a 3D vector $\textbf{q}=[q_1,q_2,q_3]^T$ to get new vector $\textbf{p}=[p_1,p_2,p_3]^T$, with an angle $\theta$ and along a rotation axis $\textbf{w}=[w_1,w_2,w_3]^T, w_1^2+w_2^2+w_3^2=1$. Such a rotation is equivalent to the following
    quaternion operation:
    $$\hat{p}=\hat{w}\hat{q}\hat{w}^*$$
    where $\hat{q}= 0 + q_1i + q_2j + q_3k$, $\hat{p}= 0 + p_1i + p_2j + p_3k$ and 
    $$\hat{w}=\cos\frac{\theta}{2}+\sin\frac{\theta}{2}(w_1i + w_2j + w_3k)$$
\end{frame}

\subsection{Real-valued CNNs and their extensions}
\begin{frame}{Related works}{Real-valued CNNs and their extensions}
Convolutional neural network is one of the most successful models in many vision tasks. CNNs have achieved encouraging results in many field like digit recognition, image classification, low-level vision.
\\~\\
ome efforts have been made to extend real-valued neural networks to other number fields. Complex-valued neural networks have been built and proved to have advantage on generalization ability and can be more easily optimized.
\\~\\
In Deep quaternion networks, a deep quaternion
network is proposed. However, its convolution simply replaces the real multipli-
cations with quaternion ones, and its quaternion kernel is not further parame-
terized. Our proposed quaternion convolution, however, is physically-meaningful
for color image processing tasks.

\end{frame}

\section{Proposed Quaternion CNNs}
\subsection{Quaternion convolution layers}
\begin{frame}{Proposed Quaternion CNNs}{Quaternion convolution layers}
Focusing on color image representation, our quaternion CNN treats a color image as a 2D pure quaternion matrix, denoted as $\hat{A}=[\hat{a}_{nn'}]\in \mathbb{H}^{N\times N}$ where $N$ represents the size of the image. In particular, the quaternion matrix $\hat{A}$ is 
$$\hat{A}=\mathbf{0}+\textbf{R}i+\textbf{G}j+\textbf{B}k,$$ where $\textbf{R},\textbf{G},\textbf{B}\in \mathbb{R}^{N\times N}$ represent red, green and blue channels, respectively.

\end{frame}

\begin{frame}{Proposed Quaternion CNNs}{Quaternion convolution layers}
    
    
\end{frame}







\end{document}
