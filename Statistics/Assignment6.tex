\documentclass{article}
\usepackage{setspace}
\usepackage{amsmath}
\usepackage{amssymb}
\usepackage{amsthm}
\usepackage{amsfonts}
\usepackage{graphicx} 
\usepackage{float} 
\usepackage{fancyhdr}                                
\usepackage{lastpage}        
\usepackage{textcomp}                               
\usepackage{layout}   
\usepackage{subfigure} 
\usepackage{geometry}
\geometry{a4paper,scale=0.8}
\pagestyle{fancy}  
\lhead{ZHANG HUAKANG}
\chead{Assignment 6} 
\rhead{DB92760} 
\renewcommand{\baselinestretch}{1.05}
\title{Assignment 6 of CISC 1006}
\author{ZHANG HUAKANG \\ DB92760 \\ \\ Computer Science, \\Faculty of Science and Technology}
\begin{document}
    \maketitle
    \section{}
        \paragraph{
            It easy to know that $X\sim Poission$ where $\lambda=3$.
            $$P(X=x)=\frac{\lambda^x}{x!}e^{-\lambda}$$
        }
        \subsection{}
            \paragraph{
                \begin{equation*}
                    \begin{split}
                        P(X=5)=&\frac{3^5}{5!}e^{-3}\\
                            \approx&0.1008\\
                    \end{split}
                \end{equation*}
            }
        \subsection{}
            \paragraph{
                \begin{equation*}
                    \begin{split}
                        P(X<3)=&\sum_{x=0}^2 P(X=x)\\
                            =&\sum_{x=0}^2 \frac{\lambda^x}{x!}e^{-\lambda}\\
                            \approx&0.4232
                    \end{split}
                \end{equation*}
            }
        \subsection{}
            \paragraph{
                \begin{equation*}
                    \begin{split}
                        P(X\geq 2)&=1-P(X\leq 1)\\
                            =&1-\sum_{x=0}^1 \frac{3^x}{x!}e^{-3}\\
                            \approx&0.8008\\
                    \end{split}
                \end{equation*}
            }
    \section{}
        \paragraph{
            It easy to know that $x~Poission$ where $\lambda=5$.
            $$P(X=x)=\frac{\lambda^x}{x!}e^{-\lambda}$$
        }
        \subsection{}
            \paragraph{
                \begin{equation*}
                    \begin{split}
                        P(X>5)=&1-P(X\leq 5)\\
                            =&1-\sum_{x=0}^5\frac{5^x}{x!}e^{-5}\\
                            \approx&0.3840\\
                    \end{split}
                \end{equation*}
            }
        \subsection{}
            \paragraph{
                \begin{equation*}
                    \begin{split}
                        P(\text{3 of next 4 days})=&C_4^3 P(X>5)^3 P(X\leq 5)\\
                            \approx&0.0349\\
                    \end{split}
                \end{equation*}
            }
        \subsection{}
        \paragraph{
            \begin{equation*}
                \begin{split}
                    P(\text{The first time in April on April 5th})=&P(X>5) P(X\leq 5)^4\\
                        \approx&0.0553\\
                \end{split}
            \end{equation*}
        }
    \section{}
        \subsection{}
            \paragraph{
                Using Binomial distribution:
                \begin{equation*}
                    \begin{split}
                        P(X<5|\text{In 2000 people})=&\sum_{x=0}^4C_{2000}^x 0.002^x(1-0.002)^{2000-x}\\
                            \approx&0.6288\\
                    \end{split}
                \end{equation*}
                Using Poission Approximation:
                \begin{equation*}
                    \begin{split}
                        \lambda=&np\\
                            =&2000\times 0.002\\
                            =&4\\
                        P(X=x)=&\frac{4^x}{x!}e^{-4}\\
                        P(X<5)=&\sum_{x=0}^4\frac{4^x}{x!}e^{-4}\\
                            \approx&0.6288\\
                    \end{split}
                \end{equation*}
            }
        \subsection{}
            \paragraph{
                Using Binomial distribution:
                \begin{equation*}
                    \begin{split}
                        P(X=x)=&C_{2000}^x 0.002^x(1-0.002)^{2000-x}\\
                    \end{split}
                \end{equation*}
                Using Poission Approximation:
                \begin{equation*}
                    \begin{split}
                        P(X=x)=&\frac{4^x}{x!}e^{-4}\\
                    \end{split}
                \end{equation*}
            }
            \subsubsection{}
                \paragraph{
                    Using Binomial distribution:
                    \begin{equation*}
                        \begin{split}
                            \mu=&\sum_{x=0}^{2000}xP(X=x)\\
                                =&\sum_{x=0}^{2000}xC_{2000}^x0.002^x(1-0.002)^{2000-x}\\
                                =&3.99\dot{9}\\
                                \approx&4.0000\\
                        \end{split}
                    \end{equation*}
                    Using Poission Approximation:
                    \begin{equation*}
                        \begin{split}
                            &\text{By definition:}\\
                            &\mu=4.000\\
                        \end{split}
                    \end{equation*}
                }
            \subsubsection{}
                \paragraph{
                    Chebyshev's inequality:
                    $$
                    P(|X-\mu|\geq k\sigma)\leq \frac{1}{k^2}
                    $$
                    and
                    $$
                    P(|X-\mu|\leq k\sigma)\geq 1-\frac{1}{k^2}.
                    $$
                    \begin{equation*}
                        \begin{split}
                            X\geq 1500\\
                            |X-\mu|>1496\\
                            \sigma^2=&\lambda\\
                                =&4\\
                            \sigma=&2\\
                            1496=&k\sigma\\
                            k=&748\\
                            &\text{Thus}\\
                            P(|X-\mu|\geq 748\sigma)\leq& \frac{1}{748^2}\\
                                =&\frac{1}{559504}\\
                                \approx&1.787\times 10^{-6}\\
                        \end{split}
                    \end{equation*}
                }
    \section{}
        \subsection{}
            \paragraph{
                \begin{equation*}
                    \begin{split}
                        P(X=4;\lambda=6)=&\frac{6^4}{4!}e^{-6}\\
                            \approx&0.1339\\
                    \end{split}
                \end{equation*}
            }
        \subsection{}
            \paragraph{
                \begin{equation*}
                    \begin{split}
                        P(X\geq 4;\lambda=6)=&1-P(X<3;\lambda=6)\\
                            =&1-\sum_{x=0}^3\frac{6^x}{x!}e^{-6}\\
                            \approx&1-0.1512\\
                            =&0.8488\\
                    \end{split}
                \end{equation*}
            }
        \subsection{}
            \paragraph{
                \begin{equation*}
                    \begin{split}
                        P(X\geq 75;\lambda=6\times12)=&1-P(X<75;\lambda=72)\\
                            =&1-\sum_{x=0}^{75} \frac{72^x}{x!}e^{-72}\\
                            \approx&1-0.6227\\
                            =&0.3773\\
                    \end{split}
                \end{equation*}
            }
    \section{}
        \subsection{}
            \paragraph{
                Let $X$ be the number of defective component.
                \begin{equation*}
                    \begin{split}
                        P(X=15)=&C_{500}^{15}0.01^{15}\times(1-0.01)^{500-15}\\
                            \approx&1.4\times 10^{-4}\\
                            =&0.00014\\
                    \end{split}
                \end{equation*}
                The probability is too small so that it is impossible that $15$ components are defective. Thus the defective rate is not $1\%$
            }
        \subsection{}
            \paragraph{
                \begin{equation*}
                    \begin{split}
                        P(X=3)=&C_{500}^3 0.01^3\times(1-0.01)^{500-3}\\
                            \approx&0.1402
                    \end{split}
                \end{equation*}
            }
        \subsection{}
            \paragraph{
                If the defective rate is $1\%$. Let $X$ be the number of defective component. Then $$X\sim Poission$$ where $\lambda=np=5$
                \begin{equation*}
                    \begin{split}
                        P(X=15)=&\frac{5^{15}}{15!}e^{-5}\\
                            \approx&0.00016
                    \end{split}
                \end{equation*}
                The probability is too small so that it is impossible that $15$ components are defective. Thus the defective rate is not $1\%$
            }
        \subsection{}
            \paragraph{
                \begin{equation*}
                    \begin{split}
                        P(X=3)=&\frac{5^{3}}{3!}e^{-5}\\
                            \approx&0.1404
                    \end{split}
                \end{equation*}
            }
    \section{}
        \paragraph{
            Let $X$ be the numbere of yares that the electrial switch work. 
            \begin{equation*}
                \begin{split}
                    P(X=x)=&\frac{1}{2}e^{-\frac{x}{2}}, (X\geq 0)
                \end{split}
            \end{equation*}
        }
        \subsection{}
            \paragraph{
                \begin{equation*}
                    \begin{split}
                        P(X\leq 1)=&\int_{-\infty}^1 P(X=x)dx\\
                            =&0+\int_0^1 \frac{1}{2}e^{-\frac{x}{2}}dx\\
                            =&1-e^{-\frac{1}{2}}\\
                            \approx&0.3935
                    \end{split}
                \end{equation*}
            }
        \subsection{}
            \paragraph{
                \begin{equation*}
                    \begin{split}
                        P=&\sum_{x=0}^{30}C_{100}^xP(X\leq 1)^x (1-P(X\leq 1))^{100-x}\\
                            \approx&0.0335
                    \end{split}
                \end{equation*}
            }
    \section{}
        \subsection{}
            \paragraph{
                The response time $X$ is exponential distributions, where it mean is $\theta=3$.
                $$P(X=x)=\frac{1}{3}e^{-\frac{x}{3}}, (X>0)$$
                and $P(X=x)=0$ elsewhere
                \begin{equation*}
                    \begin{split}
                        P(X>5)=&\int_{5}^{\infty}\frac{1}{3}e^{-\frac{x}{3}}dx\\
                            =&e^{-\frac{5}{3}}\\
                            \approx&0.1889\\
                    \end{split}
                \end{equation*}
            }
        \subsection{}
            \paragraph{
                \begin{equation*}
                    \begin{split}
                        P(X>10)=&\int_{10}^{\infty}\frac{1}{3}e^{-\frac{x}{3}}dx\\
                            =&e^{-\frac{10}{3}}\\
                            \approx&0.0357\\
                    \end{split}
                \end{equation*}
            }
    \section{}
        \subsection{}
            \paragraph{
                Exponrntial Distributions
                $$\mu={2}$$
                $$var={4}$$
            }
        \subsection{}
            \paragraph{
                \begin{equation*}
                    \begin{split}
                        P(X\geq 20)=&\int_{20}^\infty P(X=x)dx\\
                            =&\int_{20}^\infty \frac{1}{2}e^{-\frac{x}{2}}dx\\
                            =&e^{-10}\\
                            \approx&4.540\times 10^{-5}\\
                    \end{split}
                \end{equation*}
            }
        \subsection{}
            \paragraph{
                \begin{equation*}
                    \begin{split}
                        P(0<X<10)=&\int_{0}^{10}P(X=x)dx\\
                            =&\int_0^{10}\frac{1}{2}e^{-\frac{x}{2}}dx\\
                            =&1-e^{-5}\\
                            \approx&0.9933\\
                    \end{split}
                \end{equation*}
            }
        \subsection{}
            \paragraph{
                \begin{equation*}
                    \begin{split}
                        P(X\geq 2 )=&\int_{2}^\infty P(X=x)dx\\
                            =&\int_{2}^\infty \frac{1}{2}e^{-\frac{x}{2}}dx\\
                            =&e^{-1}\\
                            \approx&0.3679\\
                    \end{split}
                \end{equation*}
            }
        \subsection{}
            \paragraph{
                Let the number of messages per hour be $X$, then we can know that $X\sim Poission$ where $$\lambda=60\times \frac{1}{\beta}=30$$
                Thus,its variance $$var=30$$
            }
\end{document}