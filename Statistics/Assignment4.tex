\documentclass{article}
\usepackage{setspace}
\usepackage{amsmath}
\usepackage{amssymb}
\usepackage{amsthm}
\usepackage{amsfonts}
\usepackage{graphicx} 
\usepackage{float} 
\usepackage{fancyhdr}                                
\usepackage{lastpage}        
\usepackage{textcomp}                               
\usepackage{layout}   
\usepackage{subfigure} 
\usepackage{geometry}
\geometry{a4paper,scale=0.8}
\pagestyle{fancy}  
\lhead{ZHANG HUAKANG}
\chead{Assignment 4} 
\rhead{DB92760} 
\renewcommand{\baselinestretch}{1.05}
\title{Assignment 4 of CISC 1006}
\author{ZHANG HUAKANG \\ DB92760 \\ \\ Computer Science, \\Faculty of Science and Technology}
\begin{document}
    \maketitle
    \section{}
        The Mathematical Expectation of loos
        \begin{equation*}
            \begin{split}
                \mathbb{E}[X]=&200000\times(1\times 0.02+0.5\times 0.01+0.25\times 0.1)\\
                    =&\$19,000
            \end{split}
        \end{equation*}
        Thus the insurance company should charge
        $$\mathbb{E}[X]+\$500=\$19,500$$
    \section{}
        \subsection{}
            \begin{equation*}
                \begin{split}
                    \mathbb{E}[X]=&\int_{-\infty}^\infty xf(x)dx\\
                        =&\int_{0}^1 x^2dx+\int_1^2 x(2-x)dx\\
                        =&1\\
                \end{split}
            \end{equation*}
        \subsection{}
            \begin{equation*}
                \begin{split}
                    var(X)=&\mathbb{E}[X^2]-\mathbb{E}[X]^2\\
                        =&\int_{0}^1 x^3dx+\int_1^2 x^2(2-x)dx-1\\
                        =&\frac{7}{6}-1\\
                        =&\frac{1}{6}
                \end{split}
            \end{equation*}
    \section{}
        \begin{equation*}
            \begin{split}
                \mathbb{E}[X]=&\int_{-\infty}^\infty xf(x)dx\\
                    =&\int_0^\infty x\frac{1}{2000}e^{-\frac{x}{2000}}dx\\
                    =&\lim_{t\rightarrow \infty}-e^{-\frac{x}{2000}}(x+2000)|_0^t\\
                    =&2000
            \end{split}
        \end{equation*}
    \section{}
        \subsection{}
            \begin{equation*}
                \begin{split}
                    \mathbb{E}[Y]=&\int_{-\infty}^\infty yf(y)dy\\
                        =&\int_0^1 5y(1-y)^4\\
                        =&\frac{1}{6}
                \end{split}
            \end{equation*}
        \subsection{}
            \begin{equation*}
                \begin{split}
                    \mathbb{P}(Y>\frac{1}{6})=&\int_\frac{1}{6}^\infty f(y)dy\\
                        =&\int_\frac{1}{6}^1 5(1-y)^4dy\\
                        =&\frac{3125}{7776}\\
                        \approx& 0.4019
                \end{split}
            \end{equation*}
    \section{}
        \subsection{}
            \begin{equation*}
                \begin{split}
                    \mathbb{E}[X]=&\sum_{x=-\infty}^\infty xf(x)\\
                        =&\sum_{x=2}^6 xf(x)\\
                        =&2\times 0.01+3\times 0.25\\
                        &+4\times 0.4+5\times0.3+6\times0.04\\
                        =&\frac{411}{100}\\
                        =&4.11\\
                    var(x)=&\mathbb{E}[X^2]-\mathbb{E}[X]^2\\
                        =&\sum_{x=-\infty}^\infty x^2f(x)-\sum_{x=-\infty}^\infty xf(x)\\
                        =&\sum_{x=2}^6 x^2f(x)-\sum_{x=2}^6 xf(x)\\
                        =&\frac{1763}{100}-(\frac{411}{100})^2\\
                        =&\frac{7379}{10000}\\
                        =&0.7379\\
                \end{split}
            \end{equation*}
        \subsection{}
            \begin{equation*}
                \begin{split}
                    \mathbb{E}[Z]=&\mathbb{E}[3X-2]\\
                        =&3\mathbb{E}[X]-2\\
                        =&\frac{1033}{100}\\
                        =&10.33\\
                    var(Z)=&var(3X-2)\\ 
                        =&9var(X)\\
                        =&6.6411\\
                \end{split}
            \end{equation*}
    \section{}
        \subsection{}
            \begin{equation*}
                \begin{split}
                    \mathbb{E}[Y]=&\mathbb{E}[3X-2]\\
                        =&\int_{-\infty}^\infty (3x-2)f(x)dx\\
                        =&\int_0^\infty (3x-2)(\frac{1}{4}e^{-\frac{x}{4}})\\
                        =&-e^{-\frac{x}{4}}(3x+10)|_0^\infty\\
                        =&10\\
                    var(Y)=&\mathbb{E}[Y^2]-\mathbb{E}[Y]^2\\
                        =&\int_0^\infty (3x-2)^2(\frac{1}{4}e^{-\frac{x}{4}})-100\\
                        =&-e^{-\frac{x}{4}}(9x^2+60x+244)_0^\infty-100\\
                        =&244-100\\
                        =&144
                \end{split}
            \end{equation*}
        \subsection{}
            % It is easy to know that $X$ is a random variable with Poisson disyribution. Thus
            \begin{equation*}
                \begin{split}
                    \mathbb{E}[X]=&\int_{\infty}^\infty xf(x)dx\\
                        =&\int_0^\infty \frac{x}{4}e^{-\frac{x}{4}}\\
                        =&-e^{-\frac{x}{4}}(x+4)_0^\infty\\
                        =&4\\
                    var(X)=&\mathbb{E}[X^2]=\mathbb{E}[X]^2\\
                        =&\int_0^\infty \frac{x^2}{4}e^{-\frac{x}{4}}-16\\
                        =&-e^{-\frac{x}{4}}(x^2+8x+32)_0^{\infty}-16\\
                        =&32-16\\
                        =&16\\
                    \mathbb{E}[Y]=&3\mathbb{E}[X]-2\\
                    var(Y)=&3^2var(X)\\
                \end{split}
            \end{equation*}
\end{document}