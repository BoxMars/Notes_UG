\documentclass{article}
\usepackage{setspace}
\usepackage{amsmath}
\usepackage{amssymb}
\usepackage{amsthm}
\usepackage{graphicx} 
\usepackage{float} 
\usepackage{fancyhdr}                                
\usepackage{lastpage}        
\usepackage{siunitx} 
\usepackage{textcomp}                               
\usepackage{layout}   
\usepackage{subfigure} 
\pagestyle{fancy}  
\lhead{ZHANG HUAKANG}
\chead{Assignment 3} 
\rhead{DB92760} 
\title{Assignment 3 of CISC 3018}
\author{ZHANG HUAKANG}
\begin{document}
    \maketitle
    \section{}
        \paragraph{}
        \begin{itemize}
            \item File-based Virtual Storage provides the clients the data through directory trees, folders and indivual files which is easy to lead the signle-path issue.
            \item Block-based Virtual Storage will break up data into blocks and stroing the blocks into separate pieces, each with a unique identifier. When a client or application requests data from a block storage system, the underlying storage system reassembles the data blocks and presents the data to the client or application.
            \item Object-based Virtual Storage will break data file into pieces called objects, and the storing those objects in a single repository. Each object has an unique ID, and comparing with Block-based Virtual Storage it stors metadata about the file.
        \end{itemize}
    \section{}
        \paragraph{}
        \begin{itemize}
            \item Volume: The amount of data to be stored, processed or analyzed.
            \item Velocity: The data throughput to be stored.
            \item Variety: Different data types and formats to be stored, processed or analyzed.
        \end{itemize}
    \section{}
        \subsection{}
            \begin{itemize}
                \item Replication mechanism:  Break up data into blocks and stor one block on different machines simultaneously.
                \item Erasure Coding: Break up data into blocks and use XOR operation to generate the parity check block. And stor this three blocks in different machines.
            \end{itemize}
        \subsection{}

                \begin{tabular}{llr}
                    &A1=&\  01101110\\
                    XOR\ &A2= &\  10111001\\
                    \hline
                    &Ap= &\  11010111
                \end{tabular}
        \subsection{}

                \begin{tabular}{llr}
                    &Ap=&\  11010111\\
                    XOR\ &A2=&\  10111001\\
                    \hline
                    &A1= &\  01101110
                \end{tabular}
        \subsection{}
            \paragraph{}
            Assume the size of each bolck is $M$
            With the replication mechanism, we will store two pieces of A1 and A2, the total we use is $4M$. With the Erasure Coding, we will store one piece of A1 A2 and Ap, the total we use is $3M$. Only used $75\%$ of the space we use with replication mechanism.
    \section{}
        \subsection{}
            \paragraph{}
            \begin{itemize}
                \item NameNode: Control the clients' access file, manage the Meta-data of blocks of the clients' file, and manage the file system operations.
                \item DataNode: Executing the clients' detailed tasks and performing block operations according to the instructions from the NameNode.
            \end{itemize}
        \subsection{}
            \paragraph{}
            \begin{itemize}
                \item HMaster manages all RegionServers, and stores the Metadata to RegionServers.
                \item RegionServer: Large logical tables are separated into multiple blocks and stores them in different regionserver.
            \end{itemize}
        \subsection{}
            \paragraph{}
            They both use Master/Slave architecture. 
    \section{}
        \subsection{}
            \paragraph{}
            HMaster is easy to be attack. If the HMaster have some problems, the whole system will not work.
        \subsection{}
            \paragraph{Similarity}
            They both use NoSQL database to store data.
            \paragraph{Difference}
            \begin{itemize}
                \item HBase uses Master/Salve architecture
                \item Cassandra uses Masterlees architecture.
            \end{itemize}
\end{document}