\documentclass{article}
\usepackage{amsmath}
\usepackage{amssymb}
\usepackage{amsthm}                                        
\usepackage{textcomp}     
\usepackage{xcolor}
\usepackage{listings}   
\usepackage[norelsize, linesnumbered, ruled, lined, boxed, commentsnumbered]{algorithm2e}                   
\title{Assignment 1 of CISC 3000}
\author{ZHANG HUAKANG}
\begin{document}
    \maketitle
    \section{}
    \paragraph{} I use a Hasp Map to count each number in $A$ and $B$, and the number that the count is $1$ is what we want find.

    \begin{algorithm}[H]
        \caption{}
        $H$ is a hash map.

        \For(){$i\in A$}{
            \If(){$i \in H$}{
                $H_i$+=1
            }
            \Else(){
                $H_i$=1
            }
        }

        \For(){$i\in B$}{
            \If(){$i \in H$}{
                $H_i$+=1
            }
            \Else(){
                $H_i$=1
            }
        }

        \For(){$i\in H$}{
            \If(){$H_i=1$}{
               Output: $i \in (A\cup B) \setminus  (A\cap B)$
            }
        }
    \end{algorithm}
    This algorithm is the count of each number in $A$ or $B$ is $1$. We will prove that if the count of number is 1, this number must in $(A\cup B) \setminus  (A\cap B)$
    \begin{proof}
        Since that elements in $A$ have different values and elements in $B$ also have different values, each element $E$ in $A$ or $B$ will only show one time in its array, denoted as $No_A(E)=1$ or $No_B(E)=1$. If we put the elements in two array together into array $C$ which allows duplicate, the number of each elements $No_C(E)$ will have two case. For element $E\in A\cup B$,
        \begin{itemize}
            \item[Case 1] If $E \in A$ and $E \notin B$, then $No_C(E)=1$.
            \item[Case 2] If $E \notin A$ and $E \in B$, then $No_C(E)=1$.
            \item[Case 3] If $E \in A$ and $E \in B$, then $No_C(E)=2$.
        \end{itemize}
        So, if $No_C(E)=1$, then ($E \in A$ and $E \notin B$) or ($E \notin A$ and $E \in B$), \emph{i.e.}, if $E \in (A\cup B) \setminus  (A\cap B)$
    \end{proof}
    \paragraph{Complexity} 
    The hash operation complexity is $O(1)$
    \begin{equation}
        \begin{split}
            T(|A|+|B|)=&|A|\times O(1)+|B|\times O(1)+|(A\cup B) \setminus  (A\cap B)|\times O(1)\\
                        =&|A|+|B|+|(A\cup B) \setminus  (A\cap B)|\\
                        <&|A|+|B|+|A|+|B|\\
                        =&2(|A|+|B|)\\
                        =&O(|A|+|B|)
        \end{split}
    \end{equation}
    \section{}
    \subsection*{a}
    \begin{algorithm}[H]
        \caption{}
        $p_L=1$, $p_R=1$, $count=0$.

        \While{$p_L\leq|L|$ or $p_R\leq|R|$}{
            \If{$L_{p_L}>R_{p_R}$}{
                $count+=1$

                $p_R+=1$
            }
            \Else{
                $p_L+=1$
            }
        }
        Output: $count$
    \end{algorithm}
    \paragraph{Complexity} We have a pointer for each array. \textbf{For each while-loop, there must be only one poniter can move.} They both start from the begining of the array, and stop when both of them reach to the end of the array. $p_L$ moves $|L|$ times, and $p_R$ moves $|R|$ times. For the whole while-loop, it will be executed $|L|+|R|$ times. We know that $count\in[0,|L|+|R|] $
    \begin{equation}
        \begin{split}
            T(|L|+|R|)=&2\times count +(|L|+|R|-count)\\
                    =&count+|L|+|R|\\
                    \leq& |L|+|R|+|L|+|R|\\
                    =&2(|L|+|R|)\\
                    =&O(|L|+|R|)
        \end{split}
    \end{equation}
    \subsection*{b}
    \begin{algorithm}[H]
        \SetKwProg{Function}{function}{}{end}
        \Function{merge($A:array,p:int,q:int,r:int$)}{
            $n_1=q-p+1$

            $n_2=r-q$

            let $L[1..n_1+1]$ and $R[1..n_2+1]$

            \For{$i=1$ to $n_1$}{
                $L[i]=A[p+i-1]$
            }
            \For{$i=1$ to $n_2$}{
                $R[i]=A[q+j]$
            }
            $L[n_1+1]=R[n_2+1]=\infty$
            
            $i=j=1$

            \For{$k=p$ to $r$}{
                \If{$L[i]\leq R[j]$}{
                    $A[k]=L[i]$

                    $i+=1$
                }
                \Else{
                    $A[k]=R[i]$

                    $j+=1$
                }
            }
        }
        \Function{merge-sort($A:array,p:int,r:int$)}{
            \If{$p<r$}{
                $q=\left\lfloor (p+r)/2\right\rfloor $

                merge-sort(A,p,q)

                merge-sort(A,q+1,r)

                // Count the inversion between $A[p,q]$ and $A[q+1,r]$, $O(q-p+1)$

                count-inversion(A,p,q,r) 
                            
                merge(A,p,q,r)

            }
        }
    \end{algorithm}
    \paragraph{Complexity} 
    It is easy to find that
    \begin{equation}
        \begin{split}
            T_{merge}(p,r)=&r-p+1\\
                        =&O(r-p)
        \end{split}
    \end{equation}
    \begin{equation}
        \begin{split}
            T(n)=&2\times T(\frac{n}{2})+T_{merge}(1,n)+cn\\
                =&2\times T(\frac{n}{2})+c'n\\
                =&2\log n \times c'n +c'n\\
                =&O(n\log n)
        \end{split}
    \end{equation}
    \section{}
    \begin{proof}
        $T(1),T(2), T(3)\leq c=O(1)$ and 
        \begin{equation}
            T(n)\leq T(\frac{n}{4})+T(\frac{3}{4})+cn
        \end{equation}
        when $n\geq 4$. Thus,
        \begin{equation}
            \begin{split}
                T(4)\leq &T(1)+T(3)+cn\\
                    =&2c+4c\\
                    =&6c\\
                    \leq& \frac{6c}{4\log 4} 4 \log 4\\
                    =&O(n\log n)\\
            \end{split}
        \end{equation}
        Suppose that $\forall n\in [4,k-1],T(n)=O(n\log n)$, we have
        \begin{equation}
            \begin{split}
                T(k)\leq&T(\frac{k}{4})+T(\frac{3k}{4})+ck\\
                    \leq&c_1\frac{k}{4}\log \frac{k}{4}+c_2\frac{3k}{4}\log \frac{3k}{4}+ck\\
                    =&\frac{c_1k}{4}\log\frac{k}{4}+\frac{c_23}{4}(\log\frac{k}{4}+\log3)+ck\\
                    =&\frac{c_1k+3c_2k}{4}\log\frac{k}{4}+\frac{3\log 3c_3}{4}+ck\\
                    =&O(k\log k)
            \end{split}
        \end{equation}
        Thus, $\forall n\geq 4$, $T(n)=O(n\log n)$
    \end{proof}
    \section{}
    \begin{lstlisting}
def countSort(arr:list, n:int, exp:int)->None:
    output = [0] * n 
    count = [0] * n
    for i in range(n):
        count[i] = 0
    for i in range(n):
        count[ (arr[i] // exp) % n ] += 1
    for i in range(1, n):
        count[i] += count[i - 1]
    for i in range(n - 1, -1, -1):
        output[count[ (arr[i] // exp) % n] - 1] = arr[i]
        count[(arr[i] // exp) % n] -= 1
    for i in range(n):
        arr[i] = output[i]   
if __name__ =="__main__":
    arr = [33, 1, 22, 40, 12, 45, 32]
    n = len(arr)
    countSort(arr, n, n)
    print(arr)
    \end{lstlisting}
    \paragraph{Complexity} 
    \begin{equation}
        \begin{split}
            T(n)=&n+n+n+n\\
                =&4n\\
                =&O(n)
        \end{split}
    \end{equation}
    \section{}
    \subsection*{a}
    \begin{algorithm}[H]
        \SetKwProg{Function}{function}{}{end}
        Input: A

        MergeSort(A)

        $max\_time=-1$

        $t_n=A_0$

        $t=1$

        \For{$i=1$ to $lenght(A)-1$}{
            \If{$A_i== t_n$}{
                $t+=1$
            }
            \Else{
                \If{$t\geq max_time$}{
                    $max\_time=t$
                }
                
                $t_n=A_i$

                $t=1$
            }
        }

        \If{$max\_time> length(A)$}{
            Output :Yes
        }
        \Else{
            Output :No
        }
    \end{algorithm}
    \paragraph{Correctness}
    After sort this array, the numbers that have same value will be together.  We count the number of each number in array and record the maximum value of it. If the maximum value is greater that the half of the array length, then we can say that more than half of the numbers have the same value. 
    \paragraph{Complexity}
    \begin{equation}
        \begin{split}
            T(n)=&T_{merge sort}(n)+(c_1+2\times(n-c_2)+c_3)\\
                =&O(n\log n)+O(n)\\
                =&O(n\log n)
        \end{split}
    \end{equation}
    \newpage
    \subsection*{b}
    \begin{lstlisting}
a=[2,2,2,1]
d={}
for i in a:
    if i in d:
        d[i]+=1
    else:
        d[i]=1

flag=False
for i in d:
    if d[i]>len(a)/2:
        flag=False
        print("yes")
    break
if not flag:
    print("No")  
    \end{lstlisting}
    \paragraph{Correctness}
    We count the number of each number in array and record it in a hash map. If there is a value  that is greater that the half of the array length, then we can say that more than half of the numbers have the same value.
    \paragraph{Complexity} 
    \begin{equation}
        \begin{split}
            T(n)=&n\times O(1)+n\\
                =&2n\\
                =&O(n)
        \end{split}
    \end{equation}
\end{document}