\documentclass{article}
\title{Chapter 1}
\author{Hua Kang}
\usepackage{amsfonts}
\begin{document}
    \maketitle
    \section{Set}
        \subsection{Definition Part}
            \subsubsection{Proper Subset}
                We say that a set $A$ is a \textbf{proper subset} of a set $B$ if $A \subseteq B$, but there is at least one element of $B$ that is not in $A$. In this case we sometimes write
                $$A \subset B.$$
                In short, If $A\subseteq B$and $\exists b \in B , b \notin A$, then $A\subset B$.
            
            \subsubsection{Two set is equal}
                If $A\in B$ and $B\in A$, then two set are said to be \textbf{equal}, and we write $A=B$.
            
            \subsubsection{Set Operations}
                The \textbf{union} of sets $A$ and $B$ is the set
                $$A \cup B =\{x:x\in A \textbf{or} x\in B\} .$$
                The \textbf{intersection} of the sets $A$ and $B$ is the set
                $$ A\cap B =\{x:x\in A \textbf{and}x\in B\}.$$
                The \textbf{complement of} $B$ \textbf{relative to} $A$ is the set
                $$A\backslash B=\{x:x\in A \textbf{and} x \notin B\}.$$

            \subsubsection{Empty set and disjoint}
                The set that has no elements is called the \textbf{empty set} and is denoted by the symbol  $\emptyset$. Two set $A$ and $B$ are sasid to be \textbf{disjoint} if they have no elements in common, this can be expressed by writing $A \cap B =\emptyset$
            
            \subsubsection{Infinite union or intersection}
                $$\cup _{n=1} ^{\infty} A_n=\{x:x\in A_n ,\exists n \in \mathbb{N} \}$$

                $$\cap _{n=1} ^{\infty} A_n=\{x:x\in A_n ,\forall n \in \mathbb{N} \}$$
        \subsection{Theorem Part}
            \subsubsection{De Morgan Law}
            if $A,B,C$ are sets, then
            $$A \backslash (B\cup C) = (A\backslash B)\cap (A\backslash C)$$
            $$A\backslash (B\cap C) = (A\backslash B)\cup (A\backslash C)$$

        \subsection{Other}

    \section{Function}
        \subsection{Definition Part}
            \subsubsection{Cartesian product}
                If $A$ and $B$ are noempty sets, then the \textbf{Cartesian product} $A \times B$ of $A$ and $B$ is the set of all ordered pairs $(a,b)$ with $a\in A$ and $b\in B$. That is 
                $$A\times B = \{(a,b):a\in A,b\in B\}$$

            \subsubsection{Function}
                Let $A$ and $B$ be setd. Then a \textbf{function} from $A$ to $B$ is a set $f$ of ordered pairs in $A\times B$ such that for each $a\in A$there exists a unique $b\in B$ with $(a,b)\in f$.
                \\
                In other word, if $(a,b)\in f,(a,b')\in f$, then $b=b'$.
            \subsubsection{Domain and Range}
                The set $A$ of first elements of a function $f$ is called the \textbf{domain} of $f$ and is often denoted by $D(f)$
                \\
                The set of all second elements in $f$ is called the \textbf{range} of $f$ and is often denoted by $R(f)$
                \\
                \textbf{Note that}, although $D(f)=A$, we only have $R(f)\subseteq B$
        \subsection{Theorem Part}

        \subsection{Other}
        A function $f$ from a set $A$ into a set $B$ is a rule of correspondence that assigns to each element $x$ in $A$ a uniquely determined element $f(x)$ in $B$.
        \\
        The essential condition that :
        \begin{center}
            $(a,b)\in f$ and $(a,b')\in f$ implies that $b=b'$
        \end{center}
        is sometimes called the \textit{vertical line test}.
\end{document}
