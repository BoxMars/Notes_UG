\documentclass{article}
\usepackage{setspace}
\usepackage{amsmath}
\usepackage{amssymb}
\usepackage{amsthm}
\usepackage{graphicx} 
\usepackage{float} 
\usepackage{fancyhdr}                                
\usepackage{lastpage}                                           
\usepackage{layout}   
\usepackage{subfigure} 
\pagestyle{fancy}  
\lhead{ZHANG HUAKANG}
\chead{Assignment 8} 
\rhead{DB92760} 
\renewcommand{\baselinestretch}{1.05}
\title{Assignment 8 of MATH 2003}
\author{ZHANG Huakang/DB92760}
\newenvironment{claim}[1]{\par\noindent\textit{Claim.}\space#1}{}
\begin{document}
    \maketitle

    \section{}
        \paragraph{
            By the definition of continuous,$\forall \epsilon >0, \exists \delta >0$ s.t. if $|x-x_0|<\delta$ then $|f(x)-f(x_0)|<\epsilon$ and  $|g(x)-g(x_0)|<\epsilon$ i.e., 
            \begin{equation*}
                \begin{split}
                    -\epsilon<f(x)-f(x_0)<&\epsilon\\
                    -\epsilon<g(x)-g(x_0)<&\epsilon\\
                    f(x)-f(x_0)-(g(x)-g(x_0))<&\epsilon-\epsilon\\
                    f(x)-g(x)-(f(x_0)-g(x_0))<& 0\\
                    f(x)-g(x)<& f(x_0)-g(x_0)<0\\
                    f(x)-g(x)<&0\\
                \end{split}
            \end{equation*}
            when $|x-x_0|<\delta$ which means that $x\in V_\delta(x_0)$
        }
    \section{}
        Yes.
        \begin{proof}
            Assume that $f$ is not constant. Then is must exist $x$, $y\in [0,1]$ such that $f(x)$ and $f(y)$ are rational values and $f(x)\neq f(y)$. By \textit{Bolzano’s Intermediate Value Theorem} we know that $\forall k\in (\mathbb{R}\backslash \mathbb{Q})$ satisfies $\inf\{f(x),f(y)\}<k<\sup\{f(x),f(y)\}$,there exists a point $c\in (\inf\{x,y\},\sup\{x,y\})$ such that $f(c)=k$ whcih contradicts with that $f(x)$ is rational value. Therefore, $f(x)$ is constant.
        \end{proof}
    \section{}
        \begin{proof}
            We know that $\forall x_i\in I,\exists M_i\in \mathbb{R}$, such that
            $|f(x)|\leq M_i$ where $x\in V_\delta(i)$. Therefore,$|f(x)|\leq \sup\{M_i,M_j\}$ where $x\in V_{\delta_i}(i)\cup V_{\delta_j}(j)$
            It is easy to get that
            $$I\subset \cup_{x\in I}V_\delta(x)$$ 
            When $x\in I$, then $x\in \cup_{x\in I}V_\delta(x)$. Thus $$|f(x)|\leq \sup\{M_i:i\in I,x\in V_{\delta}(i),|f(x)|\leq M_i\}$$ which means $f(x)$ is bounded in $I$.
        \end{proof}
        \begin{proof}
            Assume that $f$ is not bounded on $I$ whcih means that $\exists x_0\in I$ such that $x\rightarrow x_0, |f(x_0)|\rightarrow +\infty$. Thus, $\forall x\in V_\delta(x_0)$, there do not exist a real number $M$ such that $|f(x)|\leq M$ which contradicts with $f$ is bounded on a neighborhood $V_{\delta_x}(x), \forall x\in I$
        \end{proof}
    \section{}
        \paragraph{
            $$g(x)=\frac{1}{x},x\in(0,1)$$
        }
    \section{}
        \subsection{}
            \begin{proof}
                Let $\{a_n\}$ and $\{b_n\}$ be two sequences defined as $a_n=n+\frac{1}{n}$ and $b_n=n$. Thus $\lim an-bn=0$. But $|f(a_n)-f(b_n)|=2+\frac{1}{n^2}\geq 2$. Therefore, $f(x)=x^2$ is not uniformly continuous on $A$
            \end{proof}
        \subsection{}
            \begin{proof}
                Let $\{a_n\}$ and $\{b_n\}$ be two sequences defined as $a_n=\frac{1}{n\pi}$ and $b_n=\frac{1}{2n\pi+\frac{\pi}{2}}$. Thsu $\lim a_n-b_n=0$. But $|g(a_n)-g(b_n)|=1$. Therefore, $g(x)=\sin\frac{1}{x}$ is not uniformly continuous on $B$
            \end{proof}
    \section{}
        \begin{proof}
            $f(x)$ is continuous on $[0,a]$ for some positive constant $a$. Because $[0,1]$ is closed bounded interval, $f(x)$ is uniformly continuous on $[0,a]$. We know that $f(x)$ is uniformly continuous on $(a,+\infty)$. Then $f(x)$ is uniformly continuous on $(0,+\infty)$ 
        \end{proof}

    \section{}
        \begin{proof}
            $g_\epsilon(x)$ is uniformly continuous on $A$ which means that when $x,u\in A$ and $|x-u|<\delta(\epsilon)$, then $$|g_\epsilon(x)-g_\epsilon(u)|<\epsilon$$
            We know that
            $$|f(x)-g_\epsilon(x)|<\epsilon$$
            and 
            $$|f(u)-g_\epsilon(u)|<\epsilon$$
            We can get
            $$f(x)-g_\epsilon(x)<\epsilon$$
            and 
            $$g_\epsilon(u)-f(u)<\epsilon$$
            Thus
            $$f(x)-f(u)-(g_\epsilon(x)-g_\epsilon(u))<2\epsilon$$
            i.e.
            $$f(x)-f(u)<2\epsilon+g_\epsilon(x)-g_\epsilon(u)<3\epsilon$$
            Therefore,
            $$f(x)-f(u)<3\epsilon$$
            $$|f(x)-f(u)|<3\epsilon$$
            where $\epsilon$ is arbitrary, then $3\epsilon$ is also arbitrary.
            whcih means that $f$ is also uniformly continuous on $A$.
        \end{proof}
    \section{}
        \begin{proof}
            Assume that $f:[0,p]\rightarrow\mathbb{R}$. $[0,p]$ is closed bounded interval and $f$ is continuous, then $f([0,p])$ is also closed bounded interval. Then $f$ is bounded. and $f$ is uniformly continuous since $[0,p]$ is closed bounded interval. Because $f([0,p])=f([np.(n+1)p]), n\in\mathbb{Z}$, we can get that $f:[np,(n+1)p]\rightarrow \mathbb{R}$ is also bounded and uniformly continuous.
            Therefor $f$ is bounded and uniformly continuous on $\mathbb{R}$ 
        \end{proof}
    \section{}
        \subsection{}
            \paragraph{
                \begin{equation*}
                    \begin{split}
                        f'(c)=&\lim _{x\rightarrow c}\frac{f(x)-f(c)}{x-c}\\
                            =&\lim _{x\rightarrow c} \frac{x^3-c^3}{x-c}\\
                            =&\lim _{x\rightarrow c} x^2+xc+c^2\\
                            =&3c^2
                    \end{split}
                \end{equation*}
            }
        \subsection{}
            \paragraph{
                \begin{equation*}
                    \begin{split}
                        g'(c)=&\lim _{x\rightarrow c} \frac{g(x)-g(c)}{x-c}\\
                            =&\lim_{x\rightarrow c}\frac{\frac{1}{x}-\frac{1}{c}}{x-c}\\
                            =&\lim_{x\rightarrow c}\frac{c-x}{xc(x-c)}\\
                            =&\lim _{x\rightarrow c} -\frac{1}{cx}\\
                            =&-\frac{1}{c^2}
                    \end{split}
                \end{equation*}
            }
        \subsection{}
            \paragraph{
                \begin{equation*}
                    \begin{split}
                        h'(c)=&\lim _{x\rightarrow c} \frac{h(x)-h(c)}{x-c}\\
                            =&\lim_{x\rightarrow c}\frac{\sqrt{x}-\sqrt{c}}{x-c}\\
                            =&\lim_{x\rightarrow c}\frac{\sqrt{x}-\sqrt{c}}{(\sqrt{x}-\sqrt{c})(\sqrt{c}+\sqrt{x})}\\
                            =&\lim_{x\rightarrow c}\frac{1}{\sqrt{x}+\sqrt{c}}\\
                            =&\frac{1}{2\sqrt{c}}
                    \end{split}
                \end{equation*}
            }
        \subsection{}
            \paragraph{
                \begin{equation*}
                    \begin{split}
                        k'(c)=&\lim_{x\rightarrow c}\frac{k(x)-k(c)}{x-c}\\
                            =&\lim_{x\rightarrow c}\frac{\frac{1}{\sqrt{x}}-\frac{1}{\sqrt{c}}}{x-c}\\
                            =&\lim_{x\rightarrow c} \frac{\sqrt{c}-\sqrt{x}}{\sqrt{cx}(x-c)}\\
                            =&\lim_{x\rightarrow c}\frac{\sqrt{c}-\sqrt{x}}{\sqrt{xc}(\sqrt{x}-\sqrt{c})(\sqrt{c}+\sqrt{x})}\\
                            =&\lim_{x\rightarrow c}-\frac{1}{\sqrt{cx}(\sqrt{x}+\sqrt{c})}\\
                            =&-\frac{1}{2c\sqrt{c}}
                    \end{split}
                \end{equation*}
            }
    \section{}
        \begin{claim}
            If $f:\mathbb{R}\rightarrow \mathbb{R}$ is an even function and  has a derivative at every point, then the derivative $f'$ is an odd function.
        \end{claim}
        \begin{proof}
            \begin{equation*}
                \begin{split}
                    f'(c)=&\lim_{x\rightarrow c}\frac{f(x)-f(c)}{x-c}\\
                    f'(-c)=&\lim_{x\rightarrow -c}\frac{f(x)-f(-c)}{x+c}\\
                        =&\lim_{x\rightarrow -c}\frac{f(x)-f(c)}{x+c}\\
                        =&\lim_{x\rightarrow c}\frac{f(-x)-f(c)}{-x+c}\\
                        =&\lim_{x\rightarrow c}\frac{f(x)-f(c)}{c-x}\\
                        =&-\lim_{x\rightarrow c}\frac{f(x)-f(c)}{x-c}\\
                        =&-f'(c)
                \end{split}
            \end{equation*}
            $f'(x)$ is odd function.
        \end{proof}
        \begin{claim}
            If $f:\mathbb{R}\rightarrow \mathbb{R}$ is an odd function and  has a derivative at every point, then the derivative $f'$ is an even function.
        \end{claim}
        \begin{proof}
            \begin{equation*}
                \begin{split}
                    g'(c)=&\lim_{x\rightarrow c}\frac{g(x)-g(c)}{x-c}\\
                    g'(-c)=&\lim_{x\rightarrow -c}\frac{g(x)-g(-c)}{x+c}\\
                        =&\lim_{x\rightarrow -c}\frac{g(x)+g(c)}{x+c}\\
                        =&\lim_{x\rightarrow c}\frac{g(-x)+g(c)}{-x+c}\\
                        =&\lim_{x\rightarrow c}\frac{-g(x)+g(c)}{c-x}\\
                        =&\lim_{x\rightarrow c}\frac{g(x)-g(c)}{x-c}\\
                        =&g'(c)\\
                \end{split}
            \end{equation*}
            $f'(x)$ is even function.
        \end{proof}

    \section{}
        \begin{claim}
            $g$ is differentiable for all $x\in\mathbb{R}$
        \end{claim}
        \begin{proof}
            When $c\neq 0$
            \begin{equation*}
                \begin{split}
                    g'(c)=&\lim_{x\rightarrow c}\frac{g(x)-g(c)}{x-c}\\
                        =&\lim_{x\rightarrow c} \frac{x^2\sin\frac{1}{x^2}-c^2\sin\frac{1}{c^2}}{x-c}\\
                        =&2c\sin\frac{1}{c^2}-\frac{2}{c}\cos\frac{1}{c^2}\\
                \end{split}
            \end{equation*}
            and when $c=0$
            \begin{equation*}
                \begin{split}
                    g'(0)=&\lim_{x\rightarrow 0}\frac{g(x)-g(0)}{x-0}\\
                        =&\lim_{x\rightarrow 0}\frac{g(x)}{x}\\
                        =&\lim_{x\rightarrow 0}x\sin\frac{1}{x^2}\\
                        =&0
                \end{split}
            \end{equation*}
            Therefore, $f$ is differentiable on $\mathbb{R}$
        \end{proof}
        \begin{claim}
            The derivative $g'$ is not bounded on the interval $[-1,1]$
        \end{claim}
        \begin{proof}
            When $x\rightarrow 0$, $2c\sin \frac{1}{c^2}\rightarrow 0$, $\frac{2}{c}\rightarrow \infty$and $|\cos \frac{1}{x^2}|\leq 1$ which means that $g'$ can not be bounded on the neighborhood of $0$ and therefore $g'$ is not bounded on $[-1,1]$
        \end{proof}
    \section{}
        \begin{proof}
            \begin{equation*}
                \begin{split}
                    f'(0)=&\lim_{x\rightarrow 0}\frac{f(x)-f(0)}{x-0}\\ 
                        =&\lim_{x\rightarrow 0} \frac{x^r\sin \frac{1}{x}}{x}\\
                        =&\lim_{x\rightarrow 0} x^{r-1}\sin \frac{1}{x}\\
                \end{split}
            \end{equation*}
            When $r>1$,\\
            $$\lim_{x\rightarrow 0} x^{r-1}\sin \frac{1}{x}=0$$
            since $\lim_{x\rightarrow 0} x=0$ and $|\sin \frac{1}{x}|<1$.\\
            When $r=1$,\\
            $$\lim_{x\rightarrow 0} x^{r-1}\sin \frac{1}{x}=\lim_{x\rightarrow 0}\sin\frac{1}{x}$$
            Limit do not exist.\\
            When $0<r<1$,\\
            $$\frac{1}{x^{1-r}}$$ is not bounded on neighborhood of $0$ and therefore limit do not exist.\\
            Thus, $f'(0)$exists when $r>1$.
        \end{proof}
\end{document}

