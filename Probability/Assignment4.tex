\documentclass{article}
\usepackage{setspace}
\usepackage{amsmath}
\usepackage{graphicx} 
\usepackage{float} 
\usepackage{fancyhdr}                                
\usepackage{lastpage}                                           
\usepackage{layout}   
\usepackage{subfigure} 
\pagestyle{fancy}  
\lhead{ZHANG HUAKANG}
\chead{Assignment 4} 
\rhead{DB92760} 
\renewcommand{\baselinestretch}{1.05}
\title{Assignment 4 of MATH 2005}
\author{ZHANG Huakang/DB92760}

\begin{document}
    \maketitle

    \section{}
        \paragraph{
            \begin{equation*}
                \begin{split}
                    E(X)=&\sum _{x=-1} ^3 xf(x)\\
                        =&-1\times \frac{3}{7} +0\times \frac{2}{7}+1\times \frac{1}{7}+2\times \frac{0}{7}+3\times \frac{1}{7}\\
                        =&\frac{1}{7}
                \end{split}
            \end{equation*}
        }

    \section{}  
        \paragraph{
            \begin{equation*}
                \begin{split}
                    E(X)=&\int _{-\infty} ^\infty yh(y) dy\\
                        =&\int _ 2 ^4 \frac{1}{8} (y+1)ydy +0\\
                        =&\frac{79}{3}
                \end{split}
            \end{equation*}
        }
    
    \section{}
        \subsection{}
                \begin{equation*}
                    \begin{split}
                        E(X)=&\int _{-\infty} ^\infty xf(x) dx\\
                            =&\int _1 ^3 x\frac{1}{x \log 3} dx + 0\\
                            =&\frac{2}{\log 3}\\
                    \end{split}
                \end{equation*}

                \begin{equation*}
                    \begin{split}
                        E(X^2)=&\int _{-\infty} ^\infty x^2f(x) dx\\
                            =&\int _1 ^3 \frac{x}{\log 3} dx\\
                            =&\frac{4}{\log 3}
                    \end{split}
                \end{equation*}

                \begin{equation*}
                    \begin{split}
                        E(X^3)=&\int _{-\infty} ^\infty x^3f(x) dx\\
                            =&\int _1 ^3 \frac{x^2}{\log 3} dx\\
                            =&\frac{26}{3\log 3}
                    \end{split}
                \end{equation*}

                \begin{equation*}
                    \begin{split}
                        E(X^3+2X^2-3X+1)=&E(X^3)+2E(X^2)-3E(X)+E(1)\\
                            =&\frac{26}{3\log 3}+2\times \frac{4}{\log 3} -3\times \frac{2}{\log 3}+1\\
                            =&\frac{35}{3\log3}
                    \end{split}
                \end{equation*}

    \section{}
        \begin{equation*}
            \begin{split}
                E(\frac{X}{Y})=&\int _0 ^1 \int _0 ^y \frac{x}{y}f(x,y) dx dy\\
                    =&\int _0 ^1 \int _0 ^y \frac{x}{y^2} dx dy\\
                    =&\int _0 ^1 \frac{x^2}{2y^2}|_0 ^y dy \\
                    =&\int _0 ^1 \frac{1}{2} dy\\
                    =& \frac{1}{2} y | _0 ^1 \\
                    =& \frac{1}{2} 
            \end{split}
        \end{equation*}
            
    \section{}
        \paragraph{
            Let $\varphi(x)$ is the money he should pay us where $x$ is the number we get from a balanced die.
            \begin{equation*}
                \begin{split}
                    E(\varphi(X))=&\sum _{x=1} ^6 \varphi(x)f(x)\\
                        =&\frac{1}{6}\sum_{x=1} ^4 \varphi(x)+\frac{5}{3}\\
                        =&0\\
                \end{split}
            \end{equation*}
            Thus, we can get,
            $$\sum _{x=1} ^4 \varphi(x)=-10$$
            which means that the total money we should pay that person when we roll a $1,2,3,$ or $4$ is equal to \$10. Hence, there are many solutions for this equation. For example,
            \begin{equation*}
                \begin{split}
                    \varphi(1)=\varphi(2)=\varphi(3)&=0,\\
                    \varphi(4)&=-10.
                \end{split}
            \end{equation*}
        }
    

    \section{}
        \subsection{}
            Let 
            \begin{equation*}
                \varphi(x;n)=
                \begin{cases}
                    x-(n-x)\times 0.4 & (0\leq x\leq n)\\
                    n& (n\leq x).
                \end{cases}
            \end{equation*}
            be the profit when produce $n$ cake(s) a day.

            \subsubsection*{(a) one of the cakes}
                \paragraph{
                    \begin{equation*}
                        \begin{split}
                            E(\varphi(X;1))=&\sum _{n=0} ^5 \varphi(x;1)f(x)\\
                                        =&\frac{1}{6}(-0.4+1+1+1+1+1)\\
                                        =&\frac{23}{30}
                        \end{split}
                    \end{equation*}
                }
            \subsubsection*{ (b) two of the cakes}
            \paragraph{
                \begin{equation*}
                    \begin{split}
                        E(\varphi(X;2))=&\sum _{n=0} ^5 \varphi(x;2)f(x)\\
                                    =&\frac{1}{6}(-0.8+0.6+2+2+2+2)\\
                                    =&1.2
                    \end{split}
                \end{equation*}
            }
            \subsubsection*{  (c) three of the cakes}
            \paragraph{
                \begin{equation*}
                    \begin{split}
                        E(\varphi(X;3))=&\sum _{n=0} ^5 \varphi(x;3)f(x)\\
                                    =&\frac{1}{6}(-1.2+0.2+1.6+3+3+3)\\
                                    =&1.6
                    \end{split}
                \end{equation*}
            }
            \subsubsection*{   (d) four of the cakes}
            \paragraph{
                \begin{equation*}
                    \begin{split}
                        E(\varphi(X;4))=&\sum _{n=0} ^5 \varphi(x;4)f(x)\\
                                    =&\frac{1}{6}(-1.6-0.2+1.2+2.6+4+4)\\
                                    =&\frac{5}{3}
                    \end{split}
                \end{equation*}
            }
            \subsubsection*{    (e) five of the cakes}
            \paragraph{
                \begin{equation*}
                    \begin{split}
                        E(\varphi(X;5))=&\sum _{n=0} ^5 \varphi(x;5)f(x)\\
                                    =&\frac{1}{6}(-2-1.6+0.8+2.2+3.6+5)\\
                                    =&\frac{4}{3}
                    \end{split}
                \end{equation*}
            }
        \subsection{}
            \paragraph{
                By $6.1$ we can know that he should bake 3 cakes a day to maximize his expected profit.
            }
    
    \section{}
        \paragraph{
            \begin{equation*}
                \begin{split}
                    
                \end{split}
            \end{equation*}
        }
\end{document}