\documentclass{article}
\usepackage{setspace}
\usepackage{amsmath}
\usepackage{amstext}
\usepackage{amssymb}
\usepackage{amsthm}
\usepackage{graphicx} 
\usepackage{float} 
\usepackage{fancyhdr}                                
\usepackage{lastpage}          
\usepackage{textcomp}                               
\usepackage{layout}   
\usepackage{subfigure} 
\pagestyle{fancy}  
\lhead{ZHANG HUAKANG}
\chead{Assignment 6} 
\rhead{DB92760} 
\renewcommand{\baselinestretch}{1.05}
\title{Assignment 6 of MATH 2005}
\author{ZHANG Huakang/DB92760}

\begin{document}
    \maketitle
    \section{}
        \begin{equation*}
            \begin{split}
                \mathbb{P}[X< -\theta\log(1-p)]=&\int_{-\infty}^{-\theta\log(1-p)}f(y)dy\\
                    =&\int_0^{-\theta\log(1-p)}\frac{1}{\theta}e^{-\frac{y}{\theta}}dy\\
                    =&-e^{-\frac{y}{\theta}}|_{0}^{-\theta\log(1-p)}\\
                    =&1-(1-p)\\
                    =&p
            \end{split}
        \end{equation*}

    \section{}
        \begin{equation*}
            \begin{split}
                \mathbb{E}[X]=&\int_{-\infty}^{\infty}xf(x)dx\\
                    =&\int_{0}^{\infty}x2\alpha xe^{-\alpha x^2}dx\\
                    =&2\alpha\int_{0}^{\infty} x^2e^{-\alpha x^2}dx\\
            \end{split}
        \end{equation*}
        Let $u=x$ and $v'=xe^{-\alpha x^2}$, thus $u'=1$ and $v=-\frac{1}{2\alpha}e^{-\alpha x^2}$
        \begin{equation*}
            \begin{split}
                \int x^2e^{-\alpha x^2}dx=&-\frac{x}{2\alpha}e^{-\alpha x^2}+\int \frac{1}{2\alpha}e^{-\alpha x^2}dx\\
                    =&-\frac{x}{2\alpha}e^{-\alpha x^2}+\frac{\sqrt{\pi}erf(\sqrt{\alpha}x)}{2\sqrt{\alpha}}\\
            \end{split}
        \end{equation*}
        Thus
        \begin{equation*}
            \begin{split}
                \mathbb{E}[X]=&[-\frac{x}{2\alpha}e^{-\alpha x^2}+\frac{\sqrt{\pi}erf(\sqrt{\alpha}x)}{2\sqrt{\alpha}}]|_0^{\infty}\\
                    =&\frac{\sqrt{\pi}}{2\sqrt{\alpha}}
            \end{split}
        \end{equation*}
        And 
        \begin{equation*}
            \begin{split}
                \mathbb{E}[X^2]=&\int_{-\infty}^{\infty}x^2f(x)dx\\
                    =&\int_{0}^{\infty}x^22\alpha xe^{-\alpha x^2}dx\\
                    =&2\alpha\int_{0}^{\infty}x^3e^{-\alpha x^2}dx\\
                    =&2\alpha(-\frac{e^{-\alpha x^2 (\alpha x^2+1)}}{2\alpha^2})|_0^\infty\\
                    =&\frac{1}{2\alpha^2}
            \end{split}
        \end{equation*}
        Therefore,
        \begin{equation*}
            \begin{split}
                var(X)=&\mathbb{E}[X^2]-\mathbb{E}[X]^2\\
                    =&\frac{1}{2\alpha^2}-(\frac{\sqrt{\pi}}{2\sqrt{\alpha}})^2\\
                    =&\frac{4-\pi}{4\alpha}
            \end{split}
        \end{equation*}

    \section{}
        \subsection{}
            \paragraph{We know that $\alpha>0$ and $\beta>0$.By the definition
                \begin{equation*}
                    \begin{split}
                        \int _{-\infty} ^{\infty}f(x)=&\int _0 ^\infty kx^{\beta-1}e^{-\alpha x^\beta}dx\\
                            =&k\times (-\frac{1}{\alpha\beta})e^{-\alpha x^\beta}|_0^\infty\\
                            =&0-(-\frac{k}{\alpha\beta})\\
                            =&\frac{k}{\alpha\beta}\\
                            =&1
                    \end{split}
                \end{equation*}
                Thus,$$k=\alpha\beta$$
            }

        \subsection{}
            \paragraph{
                \begin{equation*}
                    \begin{split}
                        \mathbb{E}[X]=&\int_{-\infty} ^\infty xf(x)dx\\
                            =&\int _0 ^\infty \alpha\beta x^\beta e^{-\alpha x^\beta}dx
                    \end{split}
                \end{equation*}
                Suppose $u=\alpha x^\beta$, then $u'=\alpha\beta x^{\beta-1}$.
                \begin{equation*}
                    \begin{split}
                        \mathbb{E}[X]=&\alpha^{\frac{1}{\beta}}\int _0 ^\infty u^\frac{1}{\beta}e^{-u}du\\
                        =&\alpha^{-\frac{1}{\beta}}\Gamma(1+\beta^{-1})\\
                    \end{split}
                \end{equation*}
            }
    \section{} 
        \paragraph{
            We know that $AD=x$, $AC=\frac{a}{2}$, and $BD=a-x$. If they can form a triangle,
            \begin{equation*}
                \begin{aligned}
                    x+a-x&>&\frac{a}{2}\\
                    |x-(a-x)|&<&\frac{a}{2}\\
                \end{aligned} 
            \end{equation*}
            We have
            $$\frac{a}{4}<x<\frac{3a}{4}$$
            $$\mathbb{P}(X)=\frac{\frac{3a}{4}-\frac{a}{4}}{a}=\frac{1}{2}$$
        }
    \section{}
        \paragraph{
            Because $X$ has gamma distribution with $\alpha=80\sqrt{n}$ and $\beta=2$, we have 
            $$\mathbb{E}[X]=\alpha\beta=160\sqrt{n}$$
            Its profit
            \begin{equation*}
                \begin{split}
                    Profit=&160\sqrt{n}-8n\\
                \end{split}
            \end{equation*}
            When $Profit'=0$, $80n^{-\frac{1}{2}}-8=0$
            $$n=100$$
            its expected profit is max.
        }

    \section{}
        \begin{equation*}
            \begin{split}
                \mathbb{P}(X>12)=&1-\mathbb{P}(X<12)\\
                    =&1-\int _0 ^{12} \frac{1}{2^3\times 2!}x^2e^{-\frac{x}{2}}dx\\
                    \approx&1-0.9380\\
                    =&0.0620
            \end{split}
        \end{equation*}

    \section{}
        \subsection{}
            \begin{equation*}
                \begin{split}
                    \mathbb{P}(X<24)=&\int_0 ^{24} \frac{1}{120}e^{-\frac{1}{120}x}dx\\
                        =&e^{-\frac{x}{120}}|_0^{24}\\
                        \approx&0.1813
                \end{split}
            \end{equation*}
        \subsection{}
            \begin{equation*}
                \begin{split}
                    \mathbb{P}(X>180)=&\int_{180} ^{\infty} \frac{1}{120}e^{-\frac{1}{120}x}dx\\
                        =&1-\int_{0} ^{180} \frac{1}{120}e^{-\frac{1}{120}x}dx\\
                        \approx&0.2231
                \end{split}
            \end{equation*}
    \section{}
        The arrival per hours $X_h$ follow the Poisson distribution with 
        $$\lambda_h=\frac{\lambda}{24}=1.2$$
        \begin{equation*}
            \begin{split}
                \mathbb{P}(X_h=0)=&e^{-\lambda_h}\frac{\lambda_h^0}{0!}\\
                    =&e^{-1.2}\\
                    =&0.3012
            \end{split}
        \end{equation*}

    \section{}
        The bad chech per hours $X_h$ follow the Poisson distribution with 
        $$\lambda_h=\frac{\lambda}{5}=0.4$$
        \begin{equation*}
            \begin{split}
                \mathbb{P}(X_h>2)=&\int_2^\infty 0.4e^{-.04x}dx\\
                    =&1-(-e^{-.04x}|_0^2)\\
                    \approx&0.4493\\
            \end{split}
        \end{equation*}
    \section{}
        \begin{proof}
            \begin{equation*}
                \begin{split}
                    cov(X,Y)=&cov(X,X^2)\\
                        =&\mathbb{E}[X^3]-\mathbb{E}[X]\mathbb{X^2}\\
                        =\mu^3+3\mu\sigma^2-\mu(\sigma^2+\mu^2)\\
                        =&2\mu\sigma^2\\
                        =&0
                \end{split}
            \end{equation*}
        since $X$ is a standard normal distribution with $\mu=0$
        \end{proof}

    \section{}
        \subsection{}
            \begin{equation*}
                \begin{split}
                    \mathbb{P}(X>1.14)=&0.5-\mathbb{P}(X<1.14)\\
                        =&0.5-0.3729\\
                        =&0.1271
                \end{split}
            \end{equation*}
        \subsection{}
            \begin{equation*}
                \begin{split}
                    \mathbb{P}(X<-0.36)=&0.5-+\mathbb{P}(X<0.36)\\
                        =&0.5+0.1406\\
                        =&0.6406
                \end{split}
            \end{equation*}
        \subsection{}
            \begin{equation*}
                \begin{split}
                    \mathbb{P}(-0.40<X<-0.09)=&0.1554-0.0359\\
                        =&0.1195
                \end{split}
            \end{equation*}
        \subsection{}
            \begin{equation*}
                \begin{split}
                    \mathbb{P}(-0.58<X<1.12)=&0.2190+0.3886\\
                        =&0.5876
                \end{split}
            \end{equation*}
    \section{}
        \subsection{}  
            \begin{equation*}
                \begin{split}
                    \mathbb{P}(Z<1.33)=&0.5+0.4082\\
                        =&0.9082
                \end{split}
            \end{equation*}
        \subsection{}  
            \begin{equation*}
                \begin{split}
                    \mathbb{P}(Z<-0.79)=&0.5-0.2852\\
                        =&0.0.2148
                \end{split}
            \end{equation*}
        \subsection{}  
            \begin{equation*}
                \begin{split}
                    \mathbb{P}(0.55<Z<1.22)=&0.388-0.2088\\
                        =&0.1800
                \end{split}
            \end{equation*}
        \subsection{}  
            \begin{equation*}
                \begin{split}
                    \mathbb{P}(-1.90<Z<0.44)=&0.4713+0.1700\\
                        =&0.0.6413
                \end{split}
            \end{equation*}
    \section{}
        From the table we can get
        \subsection{}
            1.48
        \subsection{}
            -0.74
        \subsection{}
            0.55
        \subsection{}
            2.17
    \newpage
    \section{}
        From the table
        \subsection{}
            $$1.64<Z_\alpha<1.65$$
        \subsection{}
            $$Z_\alpha=1.96$$
        \subsection{}
            $$2.32<Z_\alpha<2.33$$
        \subsection{}
            $$2.57<Z_\alpha<2.58$$
    \section{}
        \subsection{}
            $$Z=\frac{16-15.40}{0.48}=1.25$$ 
            \begin{equation*}
                \begin{split}
                    \mathbb{P}(Z>1.25)=&0.5-0.3944\\
                        =&0.1056
                \end{split}
            \end{equation*}
        \subsection{}
            $$Z=\frac{14.20-15.40}{0.48}=-2.5$$ 
            \begin{equation*}
                \begin{split}
                    \mathbb{P}(Z<-2.5)=&0.5-0.4938\\
                        =&0.0062
                \end{split}
            \end{equation*}
        \subsection{}
            $$|Z_1|=|Z_2|=\frac{15.80-15.40}{0.48}=0.83$$ 
            \begin{equation*}
                \begin{split}
                    \mathbb{P}(-0.83<Z<0.83)=&2\times 0.2967\\
                        =&0.5934
                \end{split}
            \end{equation*}

\end{document}