\documentclass{article}
\title{Chapter 2}
\date{\today}
\author{Huakang}
\begin{document}
\maketitle
% \tableofcontents
\section{Sample Space}
\subsection{Experiment}
Describe any process that generates a set of data
\subsection{Definition 2.1}

The set of all possible outcomes of a statistical experment is called the 
\textbf{sample space} and is representes by symbol $S$\\Each outcomes in a sample 
space is called an element or a member of the sample space, or simply a sample point.\\
The sample space $S$, of possible outcomes when a coin is flipped, may be written \\
$$ S=\{H,T\},$$
where $H$ and $T$ correspond to heads and tails, respectively
\section{Events}
\subsection{Definition 2.2}
An \textbf{event} is a subset of a sample space
\subsection{Definition 2.3}
The complement of an event $A$ with respect to$S$ is the subset of all elements of
 $S$ that are not in $A$. We denot the complement of $A$ by symbol $A'$.


\subsection{Definition 2.4}

The intersection of two events $A$ and $B$, denote by the symbol $A \cap B$

\subsection{Definition 2.5}


Two events $A$ and $B$ are \textbf{mutually excusive}, or \textbf{disjoint},if $A\cap B=\emptyset$,that is,
if $A$ and $B$ have no elements in common 

\subsection{Definition 2.6}
The \textbf{union} of the two events $A$ and $B$, denoted by the symbol $A \cup B$,
is the event containing all the elements that belong to $A$ and $B$ or both



\section{Counting Sample Points}















\end{document}