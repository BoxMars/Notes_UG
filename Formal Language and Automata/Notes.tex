\documentclass[titlepage]{article}
\title{Notes of\\ Formal Laguage and Automata\\ CISC 3007}
\author{Box, ZHANG Huakang}
\begin{document}
    \maketitle
    \section{Basic Definitions and Properties}
        \paragraph{Alphabets}
        \begin{itemize}
            \item An alphabet is a finite set of symbols. 
            \item Usually use $\Sigma$ to represent an alphabet.
        \end{itemize}
        \subsection*{Strings}
            \paragraph{Definition}
            \begin{itemize}
                \item A string is a finite sequence of symbols feom an alphabet.
            \end{itemize}
            \paragraph{String Operations}
            \begin{itemize}
                \item Length: $|1100|=4$
                \item Prefix
                \item Suffix
                \item Substring
                \item Concarenation: $\alpha=abd,\beta=ce,\alpha\beta=abdce$
                \item Exponentiation: $\alpha=abd, \alpha^3=abdabdabd, \alpha^0=\epsilon$
                \item Reversal: $\alpha=abd, \alpha^{Rev}=dba$
                \item Power of an alphabet: $\Sigma^k$ is the set of all $k$-length strings formed by the alphabet in $\Sigma$. e.g., $\Sigma=\{a,b\}$, $\Sigma^2=\{ab,aa,bb,ba\}$, $\Sigma^0=\{\epsilon\}$
                \item Kleen Closure: $\Sigma^*=\Sigma^0\cup \Sigma^1...=\cup_{k\geq 0}\Sigma^k$
                \item Kleen Plus: $\Sigma^+=\Sigma^1\cup \Sigma^2...=\cup_{k> 0}\Sigma^k$
            \end{itemize}
        \subsection*{Languages}
            \paragraph{Definition}
            A language is a set of strin gs over an alphabet.
    \section{Finite Automata}
        \subsection*{Deterministic Finite Automata}
            A DFA is a quintuple $(Q,\Sigma,\delta,q_0,F)$ where
            \begin{itemize}
                \item $Q$ is a finite set of states
                \item $\Sigma$ is a finite input alphabet
                \item $\delta$ is the transition function mapping $Q\times \Sigma$ to $Q$
                \item $q_0$ in $Q$ is the initial state (only one)
                \item $F\subset Q$ is the set of final state(s) (zero or more)
            \end{itemize}
            \paragraph{Language of a DFA}
            Giuven a DFA $M$, the language accepted (or recognized) by $M$ is the set of all strings that start from the initial state, and reache one of the finnal states.
        \subsubsection*{Non-deterministic Finite Automata}
            For each state, zero, one or more transitions are allowed on the same input symbol.


\end{document}   